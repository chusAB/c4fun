\documentclass[10pt,a4paper]{article}

\usepackage{array}
\usepackage{tikz}
\usepackage{numprint}
\usepackage{graphicx}
\usepackage{multirow}
\usepackage{pgfplots}
\usepackage{subcaption}
\usepackage{filecontents}
\usepackage{pdflscape}
\usepackage{array}

\begin{document}

Our bench is the following :
\begin{itemize}
\item Memory loads sampling done with \texttt{perf\_event\_open}.
\item  Sampling  started  just  before and  ended  jsut  after  memory
  accesses loop.
\item  Memory  accesses  loop   variable  and  memory  accesses  stack
  variables are  register allocated  to have exactly  X loads  in each
  loop iterration where X is the loop unrolling factor.
\item According  to the Intel doc,  sampling done only for  loads with
  latency greater  than or equal to  4 clock cycles. Because  L1 cache
  hit loads  latency maybe lower  than these  4 cycles, the  number of
  samples is low compared to the total number of loads. Is this really
  true    ?      I    would     have    expected     $nb\_samples    =
  \frac{nb\_loads}{frequency}$ but it's not the case.
\end{itemize}

\begin{figure}[h]
  \centering
  \begin{tabular}[h]{|c|c|}
    \hline
    Processors       & 2x Intel Xeon 5650 (hexa core) \\
    \hline
    Architecture     & Westmere-EP (06\_2CH)          \\
    \hline
    Core frequency   & 2.66GHz                        \\
    \hline
    Hyperthreading   & Disabled                       \\
    \hline
    L1 cache size    & 32 KiB / 32 KiB                \\
     \hline
    L2 cache size    & 256 KiB                        \\
    \hline
    L3 cache size    & 12 MiB                         \\
    \hline
    Cache line size  & 64 Bytes                       \\
    \hline
    Operating system & Linux kernel 3.11              \\
    \hline
  \end{tabular}
  \caption{System configuration}
  \label{tab:config}
\end{figure}

\begin{figure}[h]
  \begin{tabular}{|c|c|c|c|c|c|c|c|}
    \hline
    L1 & L2    & L3    & L3 snoop clean & L3 snoop mod. & Remote Cache & RAM & Rem. RAM \\
    \hline
    6  & 12-19 & 40-50 &                &               & 200          &     & 330      \\
    \hline
  \end{tabular}
  \caption{Latencies for different memories in cycles}
\end{figure}

\begin{filecontents}{seq.dat}
  seq_data
\end{filecontents}

\begin{figure}[h]
  \begin{tikzpicture}
    % \pgfplotsset{
    % log basis x={2},
    % x tick label style={/pgf/number format/1000 sep=\,},
    % log base 10 number format code/.code={
    % $\pgfmathparse{10^(#1)}\pgfmathprintnumber{\pgfmathresult}$
    % }
    % }
    \begin{axis}[legend pos=north west, ymin=0, ymax=100, xlabel={memory size (MB)}, ylabel={\% of memory samples}]
      seq_plot
      seq_legend
    \end{axis}
  \end{tikzpicture}
  \caption{percentage of memory samples for different memory size in the sequential
    accesses case with prefetching enabled.}
\end{figure}

\begin{filecontents}{rand.dat}
  rand_data
\end{filecontents}

\begin{figure}[h]
  \begin{tikzpicture}
    % \pgfplotsset{
    % log basis x={2},
    % x tick label style={/pgf/number format/1000 sep=\,},
    % log base 10 number format code/.code={
    % $\pgfmathparse{10^(#1)}\pgfmathprintnumber{\pgfmathresult}$
    % }
    % }
    \begin{axis}[legend pos=north west,ymin=0,ymax=100,xlabel={memory size (MB)},ylabel={\% of memory samples}]
      rand_plot
      rand_legend
    \end{axis}
  \end{tikzpicture}
  \caption{percentage of memory samples for different memory size in the random
    accesses case with prefetching enabled.}
\end{figure}

\end{document}
